% Options for packages loaded elsewhere
\PassOptionsToPackage{unicode}{hyperref}
\PassOptionsToPackage{hyphens}{url}
\PassOptionsToPackage{dvipsnames,svgnames,x11names}{xcolor}
%
\documentclass[
  letterpaper,
  DIV=11,
  numbers=noendperiod]{scrartcl}

\usepackage{amsmath,amssymb}
\usepackage{lmodern}
\usepackage{iftex}
\ifPDFTeX
  \usepackage[T1]{fontenc}
  \usepackage[utf8]{inputenc}
  \usepackage{textcomp} % provide euro and other symbols
\else % if luatex or xetex
  \usepackage{unicode-math}
  \defaultfontfeatures{Scale=MatchLowercase}
  \defaultfontfeatures[\rmfamily]{Ligatures=TeX,Scale=1}
\fi
% Use upquote if available, for straight quotes in verbatim environments
\IfFileExists{upquote.sty}{\usepackage{upquote}}{}
\IfFileExists{microtype.sty}{% use microtype if available
  \usepackage[]{microtype}
  \UseMicrotypeSet[protrusion]{basicmath} % disable protrusion for tt fonts
}{}
\makeatletter
\@ifundefined{KOMAClassName}{% if non-KOMA class
  \IfFileExists{parskip.sty}{%
    \usepackage{parskip}
  }{% else
    \setlength{\parindent}{0pt}
    \setlength{\parskip}{6pt plus 2pt minus 1pt}}
}{% if KOMA class
  \KOMAoptions{parskip=half}}
\makeatother
\usepackage{xcolor}
\setlength{\emergencystretch}{3em} % prevent overfull lines
\setcounter{secnumdepth}{-\maxdimen} % remove section numbering
% Make \paragraph and \subparagraph free-standing
\ifx\paragraph\undefined\else
  \let\oldparagraph\paragraph
  \renewcommand{\paragraph}[1]{\oldparagraph{#1}\mbox{}}
\fi
\ifx\subparagraph\undefined\else
  \let\oldsubparagraph\subparagraph
  \renewcommand{\subparagraph}[1]{\oldsubparagraph{#1}\mbox{}}
\fi

\usepackage{color}
\usepackage{fancyvrb}
\newcommand{\VerbBar}{|}
\newcommand{\VERB}{\Verb[commandchars=\\\{\}]}
\DefineVerbatimEnvironment{Highlighting}{Verbatim}{commandchars=\\\{\}}
% Add ',fontsize=\small' for more characters per line
\newenvironment{Shaded}{}{}
\newcommand{\AlertTok}[1]{\textcolor[rgb]{0.16,0.16,0.16}{\textbf{\colorbox[rgb]{0.80,0.14,0.11}{#1}}}}
\newcommand{\AnnotationTok}[1]{\textcolor[rgb]{0.60,0.59,0.10}{#1}}
\newcommand{\AttributeTok}[1]{\textcolor[rgb]{0.84,0.60,0.13}{#1}}
\newcommand{\BaseNTok}[1]{\textcolor[rgb]{0.96,0.45,0.00}{#1}}
\newcommand{\BuiltInTok}[1]{\textcolor[rgb]{0.84,0.36,0.05}{#1}}
\newcommand{\CharTok}[1]{\textcolor[rgb]{0.69,0.38,0.53}{#1}}
\newcommand{\CommentTok}[1]{\textcolor[rgb]{0.57,0.51,0.45}{#1}}
\newcommand{\CommentVarTok}[1]{\textcolor[rgb]{0.57,0.51,0.45}{#1}}
\newcommand{\ConstantTok}[1]{\textcolor[rgb]{0.69,0.38,0.53}{\textbf{#1}}}
\newcommand{\ControlFlowTok}[1]{\textcolor[rgb]{0.80,0.14,0.11}{\textbf{#1}}}
\newcommand{\DataTypeTok}[1]{\textcolor[rgb]{0.84,0.60,0.13}{#1}}
\newcommand{\DecValTok}[1]{\textcolor[rgb]{0.96,0.45,0.00}{#1}}
\newcommand{\DocumentationTok}[1]{\textcolor[rgb]{0.60,0.59,0.10}{#1}}
\newcommand{\ErrorTok}[1]{\textcolor[rgb]{0.80,0.14,0.11}{\underline{#1}}}
\newcommand{\ExtensionTok}[1]{\textcolor[rgb]{0.41,0.62,0.42}{\textbf{#1}}}
\newcommand{\FloatTok}[1]{\textcolor[rgb]{0.96,0.45,0.00}{#1}}
\newcommand{\FunctionTok}[1]{\textcolor[rgb]{0.41,0.62,0.42}{#1}}
\newcommand{\ImportTok}[1]{\textcolor[rgb]{0.41,0.62,0.42}{#1}}
\newcommand{\InformationTok}[1]{\textcolor[rgb]{0.16,0.16,0.16}{\colorbox[rgb]{0.51,0.65,0.60}{#1}}}
\newcommand{\KeywordTok}[1]{\textcolor[rgb]{0.24,0.22,0.21}{\textbf{#1}}}
\newcommand{\NormalTok}[1]{\textcolor[rgb]{0.24,0.22,0.21}{#1}}
\newcommand{\OperatorTok}[1]{\textcolor[rgb]{0.24,0.22,0.21}{#1}}
\newcommand{\OtherTok}[1]{\textcolor[rgb]{0.41,0.62,0.42}{#1}}
\newcommand{\PreprocessorTok}[1]{\textcolor[rgb]{0.84,0.36,0.05}{#1}}
\newcommand{\RegionMarkerTok}[1]{\textcolor[rgb]{0.57,0.51,0.45}{\colorbox[rgb]{0.98,0.96,0.84}{#1}}}
\newcommand{\SpecialCharTok}[1]{\textcolor[rgb]{0.69,0.38,0.53}{#1}}
\newcommand{\SpecialStringTok}[1]{\textcolor[rgb]{0.60,0.59,0.10}{#1}}
\newcommand{\StringTok}[1]{\textcolor[rgb]{0.60,0.59,0.10}{#1}}
\newcommand{\VariableTok}[1]{\textcolor[rgb]{0.27,0.52,0.53}{#1}}
\newcommand{\VerbatimStringTok}[1]{\textcolor[rgb]{0.60,0.59,0.10}{#1}}
\newcommand{\WarningTok}[1]{\textcolor[rgb]{0.16,0.16,0.16}{\colorbox[rgb]{0.98,0.74,0.18}{#1}}}

\providecommand{\tightlist}{%
  \setlength{\itemsep}{0pt}\setlength{\parskip}{0pt}}\usepackage{longtable,booktabs,array}
\usepackage{calc} % for calculating minipage widths
% Correct order of tables after \paragraph or \subparagraph
\usepackage{etoolbox}
\makeatletter
\patchcmd\longtable{\par}{\if@noskipsec\mbox{}\fi\par}{}{}
\makeatother
% Allow footnotes in longtable head/foot
\IfFileExists{footnotehyper.sty}{\usepackage{footnotehyper}}{\usepackage{footnote}}
\makesavenoteenv{longtable}
\usepackage{graphicx}
\makeatletter
\def\maxwidth{\ifdim\Gin@nat@width>\linewidth\linewidth\else\Gin@nat@width\fi}
\def\maxheight{\ifdim\Gin@nat@height>\textheight\textheight\else\Gin@nat@height\fi}
\makeatother
% Scale images if necessary, so that they will not overflow the page
% margins by default, and it is still possible to overwrite the defaults
% using explicit options in \includegraphics[width, height, ...]{}
\setkeys{Gin}{width=\maxwidth,height=\maxheight,keepaspectratio}
% Set default figure placement to htbp
\makeatletter
\def\fps@figure{htbp}
\makeatother

% load packages
\usepackage{geometry}
\usepackage{xcolor}
\usepackage{eso-pic}
\usepackage{fancyhdr}
\usepackage{sectsty}
\usepackage{fontspec}
\usepackage{titlesec}
\usepackage{listings} % For code listings

%% Set page size with a wider right margin
\geometry{a4paper, total={170mm,257mm}, left=20mm, top=20mm, bottom=20mm, right=50mm}

%% Define colors
\definecolor{light}{HTML}{D73F09}
\definecolor{highlight}{HTML}{800080}
\definecolor{dark}{HTML}{330033}

%% Let's add the border on the right hand side 
\AddToShipoutPicture{% 
    \AtPageLowerLeft{% 
        \put(\LenToUnit{\dimexpr\paperwidth-3cm},0){% 
            \color{light}\rule{3cm}{\LenToUnit\paperheight}%
          }%
     }%
     % logo
    \AtPageLowerLeft{% start the bar at the bottom right of the page
        \put(\LenToUnit{\dimexpr\paperwidth-3.95cm},26cm){% move it to the top right
            \color{light}\includegraphics[width=5cm]{_extensions/nrennie/PrettyPDF/logo.png}
          }%
     }%
}

%% Style the page number
\fancypagestyle{mystyle}{
  \fancyhf{}
  \renewcommand\headrulewidth{0pt}
  \fancyfoot[R]{\fontsize{28}{12}\selectfont\thepage} % Increase the font size by 10pt
  \fancyfootoffset{3.65cm}
}
\setlength{\footskip}{20pt}

%% style the chapter/section fonts
\chapterfont{\color{dark}\fontsize{20}{16.8}\selectfont}
\sectionfont{\color{dark}\fontsize{20}{16.8}\selectfont}
\subsectionfont{\color{dark}\fontsize{14}{16.8}\selectfont}
\titleformat{\subsection}
  {\sffamily\Large\bfseries}{\thesection}{1em}{}[{\titlerule[0.8pt]}]
  
% left align title
\makeatletter
\renewcommand{\maketitle}{\bgroup\setlength{\parindent}{0pt}
\begin{flushleft}
  {\sffamily\huge\textbf{\MakeUppercase{\@title}}} \vspace{0.3cm} \newline
  {\Large {\@subtitle}} \newline
  {\large\@author} \newline
  {\large\today} % Include the full date here
\end{flushleft}\egroup
}
\makeatother


%% Use some custom fonts
\setsansfont{Georgia}[
    Path=_extensions/nrennie/PrettyPDF/Georgia/,
    Scale=0.9,
    Extension = .ttf,
    UprightFont=*,
    BoldFont=*b,
    ItalicFont=*i,
    ]

\setmainfont{Kievit}[
    Path=_extensions/nrennie/PrettyPDF/Kievit/,
    Scale=0.9,
    Extension = .ttf,
    UprightFont=* Regular,
    BoldFont=* Bold,
    ItalicFont=* Black Italic,
    ]
\KOMAoption{captions}{tableheading}
\makeatletter
\makeatother
\makeatletter
\makeatother
\makeatletter
\@ifpackageloaded{caption}{}{\usepackage{caption}}
\AtBeginDocument{%
\ifdefined\contentsname
  \renewcommand*\contentsname{Table of contents}
\else
  \newcommand\contentsname{Table of contents}
\fi
\ifdefined\listfigurename
  \renewcommand*\listfigurename{List of Figures}
\else
  \newcommand\listfigurename{List of Figures}
\fi
\ifdefined\listtablename
  \renewcommand*\listtablename{List of Tables}
\else
  \newcommand\listtablename{List of Tables}
\fi
\ifdefined\figurename
  \renewcommand*\figurename{Figure}
\else
  \newcommand\figurename{Figure}
\fi
\ifdefined\tablename
  \renewcommand*\tablename{Table}
\else
  \newcommand\tablename{Table}
\fi
}
\@ifpackageloaded{float}{}{\usepackage{float}}
\floatstyle{ruled}
\@ifundefined{c@chapter}{\newfloat{codelisting}{h}{lop}}{\newfloat{codelisting}{h}{lop}[chapter]}
\floatname{codelisting}{Listing}
\newcommand*\listoflistings{\listof{codelisting}{List of Listings}}
\makeatother
\makeatletter
\@ifpackageloaded{caption}{}{\usepackage{caption}}
\@ifpackageloaded{subcaption}{}{\usepackage{subcaption}}
\makeatother
\makeatletter
\@ifpackageloaded{tcolorbox}{}{\usepackage[many]{tcolorbox}}
\makeatother
\makeatletter
\@ifundefined{shadecolor}{\definecolor{shadecolor}{HTML}{000000}}
\makeatother
\makeatletter
\makeatother
\ifLuaTeX
  \usepackage{selnolig}  % disable illegal ligatures
\fi
\IfFileExists{bookmark.sty}{\usepackage{bookmark}}{\usepackage{hyperref}}
\IfFileExists{xurl.sty}{\usepackage{xurl}}{} % add URL line breaks if available
\urlstyle{same} % disable monospaced font for URLs
\hypersetup{
  pdftitle={ST557: Homework 4},
  pdfauthor={Brian Cervantes Alvarez},
  colorlinks=true,
  linkcolor={highlight},
  filecolor={Maroon},
  citecolor={Blue},
  urlcolor={highlight},
  pdfcreator={LaTeX via pandoc}}

\title{ST557: Homework 4}
\author{Brian Cervantes Alvarez}
\date{11/29/23}

\begin{document}
\maketitle
\pagestyle{mystyle}

\ifdefined\Shaded\renewenvironment{Shaded}{\begin{tcolorbox}[sharp corners, interior hidden, enhanced, boxrule=0pt, frame hidden, breakable, borderline west={3pt}{0pt}{shadecolor}]}{\end{tcolorbox}}\fi

\hypertarget{question-1}{%
\section{Question 1}\label{question-1}}

\hypertarget{part-a}{%
\subsection{Part A}\label{part-a}}

Models 1 and 2 both highlight important factors: Weight, Height, SBP,
AAI, FEV, DSST, Physact, and Atrophy.

PC1 in both models combines anthropometric and physiological factors,
while PC2 focuses on cognitive and respiratory aspects. In Model 2, PC2
includes physical activity and atrophy, and PC3 introduces more
variation related to cognitive and respiratory factors, as well as
physical activity and atrophy.

Consistent variables that I noticed between both models: Weight, Height,
SBP, AAI, FEV, and DSST consistently impact both models.

Hence, the principal components represent a mix of physical health,
cognitive function, and physiological factors. PC1 generally captures
overall health and physiological status, while subsequent PCs reveal
details such as cognitive performance and specific physical attributes.

\begin{Shaded}
\begin{Highlighting}[]
\CommentTok{\# Load the data}
\NormalTok{physio }\OtherTok{\textless{}{-}} \FunctionTok{read.csv}\NormalTok{(}\StringTok{\textquotesingle{}PhysioData.csv\textquotesingle{}}\NormalTok{)}

\CommentTok{\# Extract the correlation matrix}
\NormalTok{correlationMatrix }\OtherTok{\textless{}{-}} \FunctionTok{as.matrix}\NormalTok{(physio)}

\CommentTok{\# Principal Component Factor Analysis}

\CommentTok{\# Factors = 2 \& 3}
\NormalTok{pcaResult2 }\OtherTok{\textless{}{-}} \FunctionTok{prcomp}\NormalTok{(correlationMatrix, }\AttributeTok{rank =} \DecValTok{2}\NormalTok{)}
\NormalTok{pcaResult3 }\OtherTok{\textless{}{-}} \FunctionTok{prcomp}\NormalTok{(correlationMatrix, }\AttributeTok{rank =} \DecValTok{3}\NormalTok{)}

\CommentTok{\# Variance explained}
\FunctionTok{print}\NormalTok{(}\FunctionTok{summary}\NormalTok{(pcaResult2)}\SpecialCharTok{$}\NormalTok{importance)}
\end{Highlighting}
\end{Shaded}

\begin{verbatim}
                             PC1       PC2                              
Standard deviation     0.6530516 0.4725061 0.3382827 0.3050721 0.2968584
Proportion of Variance 0.3552700 0.1859800 0.0953300 0.0775300 0.0734100
Cumulative Proportion  0.3552700 0.5412500 0.6365800 0.7141100 0.7875200
                                                                        
Standard deviation     0.2743533 0.2351406 0.2144054 0.1907133 0.1822645
Proportion of Variance 0.0627000 0.0460600 0.0382900 0.0303000 0.0276700
Cumulative Proportion  0.8502200 0.8962800 0.9345700 0.9648700 0.9925500
                                              
Standard deviation     0.09459717 3.361295e-17
Proportion of Variance 0.00745000 0.000000e+00
Cumulative Proportion  1.00000000 1.000000e+00
\end{verbatim}

\begin{Shaded}
\begin{Highlighting}[]
\FunctionTok{print}\NormalTok{(}\FunctionTok{summary}\NormalTok{(pcaResult3)}\SpecialCharTok{$}\NormalTok{importance)}
\end{Highlighting}
\end{Shaded}

\begin{verbatim}
                             PC1       PC2       PC3                    
Standard deviation     0.6530516 0.4725061 0.3382827 0.3050721 0.2968584
Proportion of Variance 0.3552700 0.1859800 0.0953300 0.0775300 0.0734100
Cumulative Proportion  0.3552700 0.5412500 0.6365800 0.7141100 0.7875200
                                                                        
Standard deviation     0.2743533 0.2351406 0.2144054 0.1907133 0.1822645
Proportion of Variance 0.0627000 0.0460600 0.0382900 0.0303000 0.0276700
Cumulative Proportion  0.8502200 0.8962800 0.9345700 0.9648700 0.9925500
                                              
Standard deviation     0.09459717 3.361295e-17
Proportion of Variance 0.00745000 0.000000e+00
Cumulative Proportion  1.00000000 1.000000e+00
\end{verbatim}

\begin{Shaded}
\begin{Highlighting}[]
\CommentTok{\# Loadings for factors  = 2 \& 3}
\NormalTok{loadings2 }\OtherTok{\textless{}{-}}\NormalTok{ pcaResult2}\SpecialCharTok{$}\NormalTok{rotation}
\NormalTok{loadings3 }\OtherTok{\textless{}{-}}\NormalTok{ pcaResult3}\SpecialCharTok{$}\NormalTok{rotation}
\FunctionTok{print}\NormalTok{(loadings2)}
\end{Highlighting}
\end{Shaded}

\begin{verbatim}
                 PC1         PC2
weight  -0.381143834  0.08752747
height  -0.530688629  0.08658642
physact  0.009944638 -0.10659706
ldl      0.273985249 -0.06137024
alb      0.003366355 -0.03627835
crt     -0.324579480  0.31786338
plt      0.402387871 -0.04795723
sbp      0.161546655  0.51069170
aai     -0.144510926 -0.55164730
fev     -0.426639096 -0.15666626
dsst    -0.005095845 -0.44008213
atrophy -0.040315746  0.28587062
\end{verbatim}

\begin{Shaded}
\begin{Highlighting}[]
\FunctionTok{print}\NormalTok{(loadings3)}
\end{Highlighting}
\end{Shaded}

\begin{verbatim}
                 PC1         PC2         PC3
weight  -0.381143834  0.08752747 -0.11087595
height  -0.530688629  0.08658642  0.00798927
physact  0.009944638 -0.10659706  0.72758107
ldl      0.273985249 -0.06137024 -0.27710141
alb      0.003366355 -0.03627835 -0.29683779
crt     -0.324579480  0.31786338 -0.03686026
plt      0.402387871 -0.04795723  0.03251985
sbp      0.161546655  0.51069170  0.21349093
aai     -0.144510926 -0.55164730  0.04151764
fev     -0.426639096 -0.15666626  0.09280287
dsst    -0.005095845 -0.44008213 -0.26487373
atrophy -0.040315746  0.28587062 -0.40605645
\end{verbatim}

\newpage{}

\hypertarget{part-b}{%
\subsection{Part B}\label{part-b}}

\begin{Shaded}
\begin{Highlighting}[]
\CommentTok{\# Residual Matrix for factors 2 \& 3}
\NormalTok{residualMatrix2 }\OtherTok{\textless{}{-}}\NormalTok{ correlationMatrix }\SpecialCharTok{{-}}\NormalTok{ (pcaResult2}\SpecialCharTok{$}\NormalTok{rotation }\SpecialCharTok{\%*\%} \FunctionTok{t}\NormalTok{(pcaResult2}\SpecialCharTok{$}\NormalTok{rotation))}
\NormalTok{residualMatrix3 }\OtherTok{\textless{}{-}}\NormalTok{ correlationMatrix }\SpecialCharTok{{-}}\NormalTok{ (pcaResult3}\SpecialCharTok{$}\NormalTok{rotation }\SpecialCharTok{\%*\%} \FunctionTok{t}\NormalTok{(pcaResult3}\SpecialCharTok{$}\NormalTok{rotation))}
\FunctionTok{print}\NormalTok{(residualMatrix2)}
\end{Highlighting}
\end{Shaded}

\begin{verbatim}
              weight       height     physact          ldl          alb
weight   0.847068320  0.337911425 -0.01491694  0.113370143  0.051196513
height   0.337911425  0.710872372  0.07944245 -0.005835126  0.093126964
physact -0.014916945  0.079442450  0.98853817 -0.041023262  0.010863053
ldl      0.113370143 -0.005835126 -0.04102326  0.921165776  0.121384656
alb      0.051196513  0.093126964  0.01086305  0.121384656  0.998672549
crt      0.102458395  0.165083863  0.01078488 -0.022917696  0.056908832
plt      0.008118457 -0.077065852 -0.01812162  0.083593055 -0.067389756
sbp      0.027333323 -0.033429683  0.05501439 -0.043139441  0.001274631
aai      0.080591892  0.044072893  0.02015585 -0.052790749  0.011499737
fev      0.187360175  0.364181511  0.08970387  0.043428568  0.061224563
dsst     0.093851306  0.056383103 -0.07861906 -0.024805326  0.052820262
atrophy  0.024497447  0.075383306 -0.05142430  0.009867980  0.058280743
                 crt          plt          sbp         aai         fev
weight   0.102458395  0.008118457  0.027333323  0.08059189  0.18736018
height   0.165083863 -0.077065852 -0.033429683  0.04407289  0.36418151
physact  0.010784881 -0.018121619  0.055014394  0.02015585  0.08970387
ldl     -0.022917696  0.083593055 -0.043139441 -0.05279075  0.04342857
alb      0.056908832 -0.067389756  0.001274631  0.01149974  0.06122456
crt      0.793611036 -0.008966337 -0.110924212  0.08250748  0.13413640
plt     -0.008966337  0.835784106 -0.009502598 -0.04718415 -0.01556534
sbp     -0.110924212 -0.009502598  0.713096669 -0.02479916  0.03734034
aai      0.082507476 -0.047184149 -0.024799162  0.67480185  0.08013754
fev      0.134136395 -0.015565337  0.037340339  0.08013754  0.79343476
dsst    -0.008149417 -0.002376835  0.063210054 -0.03480217  0.08420792
atrophy  0.050412261 -0.032009621 -0.078916828  0.06984529  0.01739542
                dsst     atrophy
weight   0.093851306  0.02449745
height   0.056383103  0.07538331
physact -0.078619059 -0.05142430
ldl     -0.024805326  0.00986798
alb      0.052820262  0.05828074
crt     -0.008149417  0.05041226
plt     -0.002376835 -0.03200962
sbp      0.063210054 -0.07891683
aai     -0.034802170  0.06984529
fev      0.084207924  0.01739542
dsst     0.806301754  0.13211295
atrophy  0.132112950  0.91665263
\end{verbatim}

\begin{Shaded}
\begin{Highlighting}[]
\FunctionTok{print}\NormalTok{(residualMatrix3)}
\end{Highlighting}
\end{Shaded}

\begin{verbatim}
             weight       height     physact          ldl         alb
weight   0.83477484  0.338797243  0.06575430  0.082646259  0.01828434
height   0.33879724  0.710808544  0.07362961 -0.003621288  0.09549848
physact  0.06575430  0.073629608  0.45916396  0.160590481  0.22683661
ldl      0.08264626 -0.003621288  0.16059048  0.844380583  0.03913048
alb      0.01828434  0.095498481  0.22683661  0.039130485  0.91055988
crt      0.09837148  0.165378350  0.03760371 -0.033131727  0.04596731
plt      0.01172413 -0.077325662 -0.04178245  0.092604351 -0.05773664
sbp      0.05100433 -0.035135319 -0.10031756  0.016019196  0.06464681
aai      0.08519520  0.043741197 -0.01005160 -0.041286152  0.02382374
fev      0.19764978  0.363440084  0.02218225  0.069144375  0.08877196
dsst     0.06448318  0.058499251  0.11409805 -0.098202210 -0.02580427
atrophy -0.02052445  0.078627400  0.24401469 -0.102650836 -0.06225216
                 crt          plt          sbp         aai         fev
weight   0.098371479  0.011724126  0.051004333  0.08519520  0.19764978
height   0.165378350 -0.077325662 -0.035135319  0.04374120  0.36344008
physact  0.037603709 -0.041782445 -0.100317562 -0.01005160  0.02218225
ldl     -0.033131727  0.092604351  0.016019196 -0.04128615  0.06914437
alb      0.045967313 -0.057736636  0.064646805  0.02382374  0.08877196
crt      0.792252357 -0.007767647 -0.103054881  0.08403783  0.13755713
plt     -0.007767647  0.834726565 -0.016445291 -0.04853430 -0.01858327
sbp     -0.103054881 -0.016445291  0.667518294 -0.03366280  0.01752777
aai      0.084037827 -0.048534297 -0.033662802  0.67307813  0.07628458
fev      0.137557133 -0.018583273  0.017527768  0.07628458  0.78482239
dsst    -0.017912732  0.006236819  0.119758191 -0.02380524  0.10878897
atrophy  0.035444914 -0.018804727  0.007772539  0.08670380  0.05507863
                dsst      atrophy
weight   0.064483179 -0.020524449
height   0.058499251  0.078627400
physact  0.114098052  0.244014686
ldl     -0.098202210 -0.102650836
alb     -0.025804270 -0.062252156
crt     -0.017912732  0.035444914
plt      0.006236819 -0.018804727
sbp      0.119758191  0.007772539
aai     -0.023805237  0.086703799
fev      0.108788967  0.055078627
dsst     0.736143662  0.024559264
atrophy  0.024559264  0.751770792
\end{verbatim}

\newpage{}

\hypertarget{part-c}{%
\subsection{Part C}\label{part-c}}

In Model 1, the first factor is shaped by variables such as weight,
height, physical activity physact, fev, and dsst. This factor indicates
a combination of physical activity levels, and cognitive and respiratory
elements. The second factor, in contrast, is influenced by sbp, aai, and
atrophy. It places emphasis on physiological aspects such as blood
pressure, arterial health, and the presence of atrophy in the studied
subjects.

Moving on to Model 2, the first factor shares similarities with Model 1,
being driven by weight, height, physical activity ,fev, and the dsst.
This factor continues to represent a blend of body measurements,
physical activity, and cognitive and respiratory factors. The second
factor, however, is influenced by additional variables including ldl,
crt, plt, sbp, aai, and atrophy. In other words, this factor focuses on
physiological factors such as lipid levels, blood pressure, arterial
health, and the presence of atrophy. Lastly, the third factor features
crt, plt, fev, and dsst, respectively.

Together, these models from the MLFA provide the health indicators and
factors within the studied population.

\begin{Shaded}
\begin{Highlighting}[]
\CommentTok{\# Maximum Likelihood Factor Analysis}

\CommentTok{\# Factors = 2 \& 3}
\NormalTok{mlfaResult2 }\OtherTok{\textless{}{-}} \FunctionTok{factanal}\NormalTok{(}\AttributeTok{covmat =}\NormalTok{ correlationMatrix, }\AttributeTok{factors =} \DecValTok{2}\NormalTok{, }\AttributeTok{method =} \StringTok{"ml"}\NormalTok{)}
\NormalTok{mlfaResult3 }\OtherTok{\textless{}{-}} \FunctionTok{factanal}\NormalTok{(}\AttributeTok{covmat =}\NormalTok{ correlationMatrix, }\AttributeTok{factors =} \DecValTok{3}\NormalTok{, }\AttributeTok{method =} \StringTok{"ml"}\NormalTok{)}

\CommentTok{\# Loadings for factors = 2 \& 3}
\NormalTok{loadings2 }\OtherTok{\textless{}{-}}\NormalTok{ mlfaResult2}\SpecialCharTok{$}\NormalTok{loadings}
\NormalTok{loadings3 }\OtherTok{\textless{}{-}}\NormalTok{ mlfaResult3}\SpecialCharTok{$}\NormalTok{loadings}
\FunctionTok{print}\NormalTok{(loadings2)}
\end{Highlighting}
\end{Shaded}

\begin{verbatim}

Loadings:
        Factor1 Factor2
weight   0.569         
height   0.956         
physact                
ldl     -0.159         
alb                    
crt      0.395  -0.126 
plt     -0.308         
sbp             -0.443 
aai              0.686 
fev      0.592   0.283 
dsst             0.342 
atrophy  0.132  -0.151 

               Factor1 Factor2
SS loadings      1.900   0.918
Proportion Var   0.158   0.077
Cumulative Var   0.158   0.235
\end{verbatim}

\begin{Shaded}
\begin{Highlighting}[]
\FunctionTok{print}\NormalTok{(loadings3)}
\end{Highlighting}
\end{Shaded}

\begin{verbatim}

Loadings:
        Factor1 Factor2 Factor3
weight   0.563   0.144         
height   0.945                 
physact                        
ldl     -0.235   0.967         
alb              0.151         
crt      0.407          -0.108 
plt     -0.316   0.122         
sbp                     -0.441 
aai                      0.695 
fev      0.579           0.304 
dsst                     0.339 
atrophy  0.139          -0.145 

               Factor1 Factor2 Factor3
SS loadings      1.895   1.016   0.945
Proportion Var   0.158   0.085   0.079
Cumulative Var   0.158   0.243   0.321
\end{verbatim}

\newpage{}

\hypertarget{part-d}{%
\subsection{Part D}\label{part-d}}

After some side research, it was found that the residual values are zero
(NULL), which means our factors and their connections perfectly match
what we observe in the study. Again, this is not something that happens
a lot in real-world situations. Getting a perfect match like this is
like finding a needle in a haystack.

If this is not the correct method to get the residuals, then I must
please leave me feedback with the correct R code. But I am certain this
is the method.

\begin{Shaded}
\begin{Highlighting}[]
\CommentTok{\# Residual Matrix for factors = 2 \& 3}
\NormalTok{residualMatrix2 }\OtherTok{\textless{}{-}} \FunctionTok{residuals}\NormalTok{(mlfaResult2)}
\NormalTok{residualMatrix3 }\OtherTok{\textless{}{-}} \FunctionTok{residuals}\NormalTok{(mlfaResult3)}
\FunctionTok{print}\NormalTok{(residualMatrix2)}
\end{Highlighting}
\end{Shaded}

\begin{verbatim}
NULL
\end{verbatim}

\begin{Shaded}
\begin{Highlighting}[]
\FunctionTok{print}\NormalTok{(residualMatrix3)}
\end{Highlighting}
\end{Shaded}

\begin{verbatim}
NULL
\end{verbatim}

\newpage{}

\hypertarget{part-e}{%
\subsection{Part E}\label{part-e}}

I would opt for using Principal Component Analysis for this dataset
since I prioritize interpretability and simplicity. PCA's ability to
capture maximum variance is a very useful tool. However, the choice is
influenced by the challenge of understanding the factors from Maximum
Likelihood Factor Analysis due to limited information on loadings and
residuals. This lack of clarity led me to favor PCA for its transparency
and ease of use in this context. Plus, I've used Principle Component
Analysis before on a few machine learning projects so it's more easier
for me to work with.

\newpage{}

\hypertarget{part-f}{%
\subsection{Part F}\label{part-f}}

For models with m = 2 factors, both PCA and MLFA yield similar factors,
emphasizing anthropometric, physiological, cognitive, and respiratory
aspects. Likewise the m = 3 models, yielded similar results with PCA's
PC3 aligning with MLFA's third factor, introducing additional variation
related to cognitive, respiratory, and physical health factors.

Therefore, both methods are effective, and improved interpretability
depends on one's familiarity with each tool.

\newpage{}

\hypertarget{question-2}{%
\section{Question 2}\label{question-2}}

The first canonical variate, the weights for glucose intolerance,
insulin response to oral glucose, and insulin resistance are
approximately -528.870, -2174.966, and -2383.596, just to highlight.

Now, here is my best interpretation of the canonical variables (still
rather new for me to grasp). In the context of this study, the results
should provide insights into how the variables of the primary variables,
glucose and insulin, and the secondary variables, weight and glucose
levels, are correlated in non-diabetic patients. The canonical variables
and correlations help summarize and quantify these relationships in a
way that maximizes the shared information between the two sets of
variables.

I think that is how it is? Truthfully, I do not fully understand how to
interpret the results of the canonical correlations, but it's still a
wonderful tool that I may touch upon in the near future.

\begin{Shaded}
\begin{Highlighting}[]
\CommentTok{\# Given covariance matrix}
\NormalTok{S }\OtherTok{\textless{}{-}} \FunctionTok{matrix}\NormalTok{(}\FunctionTok{c}\NormalTok{(}\FloatTok{1106.00}\NormalTok{, }\FloatTok{396.70}\NormalTok{, }\FloatTok{108.40}\NormalTok{, }\FloatTok{0.79}\NormalTok{, }\FloatTok{26.23}\NormalTok{,}
              \FloatTok{396.70}\NormalTok{, }\FloatTok{2382.00}\NormalTok{, }\FloatTok{1143.00}\NormalTok{, }\SpecialCharTok{{-}}\FloatTok{0.21}\NormalTok{, }\SpecialCharTok{{-}}\FloatTok{23.96}\NormalTok{,}
              \FloatTok{108.40}\NormalTok{, }\FloatTok{1143.00}\NormalTok{, }\FloatTok{2136.00}\NormalTok{, }\FloatTok{2.19}\NormalTok{, }\SpecialCharTok{{-}}\FloatTok{20.84}\NormalTok{,}
              \FloatTok{0.79}\NormalTok{, }\SpecialCharTok{{-}}\FloatTok{0.21}\NormalTok{, }\FloatTok{2.19}\NormalTok{, }\FloatTok{0.02}\NormalTok{, }\FloatTok{0.22}\NormalTok{,}
              \FloatTok{26.23}\NormalTok{, }\SpecialCharTok{{-}}\FloatTok{23.96}\NormalTok{, }\SpecialCharTok{{-}}\FloatTok{20.84}\NormalTok{, }\FloatTok{0.22}\NormalTok{, }\FloatTok{70.56}\NormalTok{), }\AttributeTok{nrow =} \DecValTok{5}\NormalTok{, }\AttributeTok{byrow =} \ConstantTok{TRUE}\NormalTok{)}

\CommentTok{\# Split the covariance matrix into S11, S12, S21, and S22}
\NormalTok{S11 }\OtherTok{\textless{}{-}}\NormalTok{ S[}\DecValTok{1}\SpecialCharTok{:}\DecValTok{3}\NormalTok{, }\DecValTok{1}\SpecialCharTok{:}\DecValTok{3}\NormalTok{]}
\NormalTok{S12 }\OtherTok{\textless{}{-}}\NormalTok{ S[}\DecValTok{1}\SpecialCharTok{:}\DecValTok{3}\NormalTok{, }\DecValTok{4}\SpecialCharTok{:}\DecValTok{5}\NormalTok{]}
\NormalTok{S21 }\OtherTok{\textless{}{-}}\NormalTok{ S[}\DecValTok{4}\SpecialCharTok{:}\DecValTok{5}\NormalTok{, }\DecValTok{1}\SpecialCharTok{:}\DecValTok{3}\NormalTok{]}
\NormalTok{S22 }\OtherTok{\textless{}{-}}\NormalTok{ S[}\DecValTok{4}\SpecialCharTok{:}\DecValTok{5}\NormalTok{, }\DecValTok{4}\SpecialCharTok{:}\DecValTok{5}\NormalTok{]}

\CommentTok{\# Find canonical variable}
\NormalTok{A1 }\OtherTok{\textless{}{-}}\NormalTok{ S11}\SpecialCharTok{\^{}}\NormalTok{(}\SpecialCharTok{{-}}\DecValTok{1}\SpecialCharTok{/}\DecValTok{2}\NormalTok{) }\SpecialCharTok{\%*\%}\NormalTok{ S12 }\SpecialCharTok{\%*\%}\NormalTok{ S22}\SpecialCharTok{\^{}}\NormalTok{(}\SpecialCharTok{{-}}\DecValTok{1}\NormalTok{) }\SpecialCharTok{\%*\%}\NormalTok{  S21 }\SpecialCharTok{\%*\%}\NormalTok{ S11}\SpecialCharTok{\^{}}\NormalTok{(}\SpecialCharTok{{-}}\DecValTok{1}\SpecialCharTok{/}\DecValTok{2}\NormalTok{)}
\NormalTok{A2 }\OtherTok{\textless{}{-}}\NormalTok{ S22}\SpecialCharTok{\^{}}\NormalTok{(}\SpecialCharTok{{-}}\DecValTok{1}\SpecialCharTok{/}\DecValTok{2}\NormalTok{) }\SpecialCharTok{\%*\%}\NormalTok{ S21 }\SpecialCharTok{\%*\%}\NormalTok{ S11}\SpecialCharTok{\^{}}\NormalTok{(}\SpecialCharTok{{-}}\DecValTok{1}\NormalTok{) }\SpecialCharTok{\%*\%}\NormalTok{  S12 }\SpecialCharTok{\%*\%}\NormalTok{ S22}\SpecialCharTok{\^{}}\NormalTok{(}\SpecialCharTok{{-}}\DecValTok{1}\SpecialCharTok{/}\DecValTok{2}\NormalTok{)}

\CommentTok{\# Compute canonical variables and correlations for all canonical variates}
\NormalTok{num\_canonical\_vars }\OtherTok{\textless{}{-}} \FunctionTok{min}\NormalTok{(}\FunctionTok{dim}\NormalTok{(S11)[}\DecValTok{1}\NormalTok{], }\FunctionTok{dim}\NormalTok{(S22)[}\DecValTok{1}\NormalTok{])}
\NormalTok{canonical\_variables }\OtherTok{\textless{}{-}} \FunctionTok{matrix}\NormalTok{(}\ConstantTok{NA}\NormalTok{, }\AttributeTok{nrow =} \FunctionTok{dim}\NormalTok{(S11)[}\DecValTok{1}\NormalTok{], }\AttributeTok{ncol =}\NormalTok{ num\_canonical\_vars)}
\NormalTok{canonical\_correlations }\OtherTok{\textless{}{-}} \FunctionTok{numeric}\NormalTok{(num\_canonical\_vars)}

\ControlFlowTok{for}\NormalTok{ (i }\ControlFlowTok{in} \DecValTok{1}\SpecialCharTok{:}\NormalTok{num\_canonical\_vars) \{}
\NormalTok{  eigenvectors\_A1 }\OtherTok{\textless{}{-}} \FunctionTok{eigen}\NormalTok{(A1)}\SpecialCharTok{$}\NormalTok{vectors}
\NormalTok{  eigenvectors\_A2 }\OtherTok{\textless{}{-}} \FunctionTok{eigen}\NormalTok{(A2)}\SpecialCharTok{$}\NormalTok{vectors}
  
\NormalTok{  a }\OtherTok{\textless{}{-}}\NormalTok{ eigenvectors\_A1[, i]}
\NormalTok{  b }\OtherTok{\textless{}{-}}\NormalTok{ eigenvectors\_A2[, i]}
  
\NormalTok{  U }\OtherTok{\textless{}{-}} \FunctionTok{Re}\NormalTok{(a }\SpecialCharTok{\%*\%} \FunctionTok{t}\NormalTok{(S11))  }
\NormalTok{  V }\OtherTok{\textless{}{-}} \FunctionTok{Re}\NormalTok{(b }\SpecialCharTok{\%*\%} \FunctionTok{t}\NormalTok{(S22))  }
  
\NormalTok{  lambda }\OtherTok{\textless{}{-}} \FunctionTok{eigen}\NormalTok{(A1)}\SpecialCharTok{$}\NormalTok{values[i]}
\NormalTok{  r }\OtherTok{\textless{}{-}} \FunctionTok{sqrt}\NormalTok{(lambda)}
  
\NormalTok{  canonical\_variables[, i] }\OtherTok{\textless{}{-}}\NormalTok{ U}
\NormalTok{  canonical\_correlations[i] }\OtherTok{\textless{}{-}}\NormalTok{ r}
\NormalTok{\}}

\CommentTok{\# Print canonical variables and correlations for all canonical variates}
\FunctionTok{print}\NormalTok{(}\StringTok{"Canonical Variables:"}\NormalTok{)}
\end{Highlighting}
\end{Shaded}

\begin{verbatim}
[1] "Canonical Variables:"
\end{verbatim}

\begin{Shaded}
\begin{Highlighting}[]
\FunctionTok{print}\NormalTok{(canonical\_variables)}
\end{Highlighting}
\end{Shaded}

\begin{verbatim}
          [,1]        [,2]
[1,]  -528.870  -117.57781
[2,] -2174.966 -1542.42981
[3,] -2383.596   -62.88942
\end{verbatim}

\begin{Shaded}
\begin{Highlighting}[]
\FunctionTok{print}\NormalTok{(}\StringTok{"Canonical Correlations:"}\NormalTok{)}
\end{Highlighting}
\end{Shaded}

\begin{verbatim}
[1] "Canonical Correlations:"
\end{verbatim}

\begin{Shaded}
\begin{Highlighting}[]
\FunctionTok{print}\NormalTok{(canonical\_correlations)}
\end{Highlighting}
\end{Shaded}

\begin{verbatim}
[1] 1.798948e+00 7.300048e-08
\end{verbatim}

\newpage{}

\hypertarget{question-3}{%
\section{Question 3}\label{question-3}}

\hypertarget{part-a-1}{%
\subsection{Part A}\label{part-a-1}}

\begin{Shaded}
\begin{Highlighting}[]
\FunctionTok{library}\NormalTok{(MASS)}
\FunctionTok{library}\NormalTok{(caret)}

\NormalTok{crudeOil }\OtherTok{\textless{}{-}} \FunctionTok{read.csv}\NormalTok{(}\StringTok{"CrudeOilData.csv"}\NormalTok{)}
\FunctionTok{colnames}\NormalTok{(crudeOil) }\OtherTok{\textless{}{-}} \FunctionTok{c}\NormalTok{(}\StringTok{"Population"}\NormalTok{, }
                        \StringTok{"Vanadium"}\NormalTok{, }
                        \StringTok{"Iron"}\NormalTok{, }
                        \StringTok{"Beryllium"}\NormalTok{, }
                        \StringTok{"SaturatedHydrocarbons"}\NormalTok{, }
                        \StringTok{"AromaticHydrocarbons"}\NormalTok{)}
\NormalTok{crudeOil}\SpecialCharTok{$}\NormalTok{Population }\OtherTok{\textless{}{-}} \FunctionTok{as.factor}\NormalTok{(crudeOil}\SpecialCharTok{$}\NormalTok{Population)}
\FunctionTok{head}\NormalTok{(crudeOil)}
\end{Highlighting}
\end{Shaded}

\begin{verbatim}
  Population Vanadium Iron Beryllium SaturatedHydrocarbons AromaticHydrocarbons
1          1      5.0   47      0.07                  7.06                 6.10
2          1      3.4   32      0.20                  5.82                 4.69
3          1      1.2   12      0.00                  5.54                 3.15
4          1      8.4   17      0.07                  6.31                 4.55
5          1      4.2   36      0.50                  9.25                 4.95
6          1      4.2   35      0.50                  5.69                 2.22
\end{verbatim}

\begin{Shaded}
\begin{Highlighting}[]
\NormalTok{ldaModel }\OtherTok{\textless{}{-}} \FunctionTok{lda}\NormalTok{(Population }\SpecialCharTok{\textasciitilde{}}\NormalTok{ ., }\AttributeTok{data =}\NormalTok{ crudeOil)}
\NormalTok{x0 }\OtherTok{\textless{}{-}} \FunctionTok{data.frame}\NormalTok{(}\AttributeTok{Vanadium =} \FloatTok{1.0}\NormalTok{, }
                 \AttributeTok{Iron =} \FloatTok{30.0}\NormalTok{, }\AttributeTok{Beryllium =} \FloatTok{0.07}\NormalTok{, }
                 \AttributeTok{SaturatedHydrocarbons =} \FloatTok{8.34}\NormalTok{, }
                 \AttributeTok{AromaticHydrocarbons =} \FloatTok{9.59}\NormalTok{)}
\NormalTok{classificationX0 }\OtherTok{\textless{}{-}} \FunctionTok{predict}\NormalTok{(ldaModel, }\AttributeTok{newdata =}\NormalTok{ x0)}\SpecialCharTok{$}\NormalTok{class}
\FunctionTok{print}\NormalTok{(}\FunctionTok{paste}\NormalTok{(}\StringTok{"Classification for x0: "}\NormalTok{, classificationX0))}
\end{Highlighting}
\end{Shaded}

\begin{verbatim}
[1] "Classification for x0:  1"
\end{verbatim}

\newpage{}

\hypertarget{part-b-1}{%
\subsection{Part B}\label{part-b-1}}

\begin{Shaded}
\begin{Highlighting}[]
\FunctionTok{set.seed}\NormalTok{(}\DecValTok{2392}\NormalTok{)}
\NormalTok{trainIndex }\OtherTok{\textless{}{-}} \FunctionTok{createDataPartition}\NormalTok{(crudeOil}\SpecialCharTok{$}\NormalTok{Population, }\AttributeTok{p =} \FloatTok{0.8}\NormalTok{, }\AttributeTok{list =} \ConstantTok{FALSE}\NormalTok{)}
\NormalTok{trainData }\OtherTok{\textless{}{-}}\NormalTok{ crudeOil[trainIndex, ]}
\NormalTok{testData }\OtherTok{\textless{}{-}}\NormalTok{ crudeOil[}\SpecialCharTok{{-}}\NormalTok{trainIndex, ]}

\NormalTok{ldaModel }\OtherTok{\textless{}{-}} \FunctionTok{lda}\NormalTok{(Population }\SpecialCharTok{\textasciitilde{}}\NormalTok{ ., }\AttributeTok{data =}\NormalTok{ trainData)}
\NormalTok{predictions }\OtherTok{\textless{}{-}} \FunctionTok{predict}\NormalTok{(ldaModel, }\AttributeTok{newdata =}\NormalTok{ testData)}\SpecialCharTok{$}\NormalTok{class}
\NormalTok{confMatrix }\OtherTok{\textless{}{-}} \FunctionTok{confusionMatrix}\NormalTok{(predictions, testData}\SpecialCharTok{$}\NormalTok{Population)}
\NormalTok{APER }\OtherTok{\textless{}{-}} \DecValTok{1} \SpecialCharTok{{-}}\NormalTok{ confMatrix}\SpecialCharTok{$}\NormalTok{overall[}\StringTok{"Accuracy"}\NormalTok{]}

\FunctionTok{print}\NormalTok{(}\StringTok{"Confusion Matrix:"}\NormalTok{)}
\end{Highlighting}
\end{Shaded}

\begin{verbatim}
[1] "Confusion Matrix:"
\end{verbatim}

\begin{Shaded}
\begin{Highlighting}[]
\FunctionTok{print}\NormalTok{(confMatrix)}
\end{Highlighting}
\end{Shaded}

\begin{verbatim}
Confusion Matrix and Statistics

          Reference
Prediction 1 2
         1 1 0
         2 1 7
                                          
               Accuracy : 0.8889          
                 95% CI : (0.5175, 0.9972)
    No Information Rate : 0.7778          
    P-Value [Acc > NIR] : 0.372           
                                          
                  Kappa : 0.6087          
                                          
 Mcnemar's Test P-Value : 1.000           
                                          
            Sensitivity : 0.5000          
            Specificity : 1.0000          
         Pos Pred Value : 1.0000          
         Neg Pred Value : 0.8750          
             Prevalence : 0.2222          
         Detection Rate : 0.1111          
   Detection Prevalence : 0.1111          
      Balanced Accuracy : 0.7500          
                                          
       'Positive' Class : 1               
                                          
\end{verbatim}

\begin{Shaded}
\begin{Highlighting}[]
\FunctionTok{print}\NormalTok{(}\FunctionTok{paste}\NormalTok{(}\StringTok{"Apparent Error Rate: "}\NormalTok{, APER))}
\end{Highlighting}
\end{Shaded}

\begin{verbatim}
[1] "Apparent Error Rate:  0.111111111111111"
\end{verbatim}

\newpage{}

\hypertarget{part-c-1}{%
\subsection{Part C}\label{part-c-1}}

\begin{Shaded}
\begin{Highlighting}[]
\NormalTok{ldaModel }\OtherTok{\textless{}{-}} \FunctionTok{lda}\NormalTok{(Population }\SpecialCharTok{\textasciitilde{}}\NormalTok{ ., }\AttributeTok{data =}\NormalTok{ crudeOil)}
\NormalTok{newObs }\OtherTok{\textless{}{-}} \FunctionTok{data.frame}\NormalTok{(}\AttributeTok{Vanadium =} \FloatTok{4.0}\NormalTok{, }
                             \AttributeTok{Iron =} \FloatTok{17.0}\NormalTok{, }
                             \AttributeTok{Beryllium =} \FloatTok{0.50}\NormalTok{, }
                             \AttributeTok{SaturatedHydrocarbons =} \FloatTok{5.54}\NormalTok{,}
                             \AttributeTok{AromaticHydrocarbons =} \FloatTok{3.51}\NormalTok{)}
\NormalTok{classificationX }\OtherTok{\textless{}{-}} \FunctionTok{predict}\NormalTok{(ldaModel, }\AttributeTok{newdata =}\NormalTok{ newObs)}\SpecialCharTok{$}\NormalTok{class}
\FunctionTok{print}\NormalTok{(}\FunctionTok{paste}\NormalTok{(}\StringTok{"Classification for x: "}\NormalTok{, classificationX))}
\end{Highlighting}
\end{Shaded}

\begin{verbatim}
[1] "Classification for x:  2"
\end{verbatim}



\end{document}
