% Options for packages loaded elsewhere
\PassOptionsToPackage{unicode}{hyperref}
\PassOptionsToPackage{hyphens}{url}
\PassOptionsToPackage{dvipsnames,svgnames,x11names}{xcolor}
%
\documentclass[
  11pt,
]{report}

\usepackage{amsmath,amssymb}
\usepackage{setspace}
\usepackage{iftex}
\ifPDFTeX
  \usepackage[T1]{fontenc}
  \usepackage[utf8]{inputenc}
  \usepackage{textcomp} % provide euro and other symbols
\else % if luatex or xetex
  \usepackage{unicode-math}
  \defaultfontfeatures{Scale=MatchLowercase}
  \defaultfontfeatures[\rmfamily]{Ligatures=TeX,Scale=1}
\fi
\usepackage{lmodern}
\ifPDFTeX\else  
    % xetex/luatex font selection
\fi
% Use upquote if available, for straight quotes in verbatim environments
\IfFileExists{upquote.sty}{\usepackage{upquote}}{}
\IfFileExists{microtype.sty}{% use microtype if available
  \usepackage[]{microtype}
  \UseMicrotypeSet[protrusion]{basicmath} % disable protrusion for tt fonts
}{}
\makeatletter
\@ifundefined{KOMAClassName}{% if non-KOMA class
  \IfFileExists{parskip.sty}{%
    \usepackage{parskip}
  }{% else
    \setlength{\parindent}{0pt}
    \setlength{\parskip}{6pt plus 2pt minus 1pt}}
}{% if KOMA class
  \KOMAoptions{parskip=half}}
\makeatother
\usepackage{xcolor}
\usepackage[margin=1in]{geometry}
\setlength{\emergencystretch}{3em} % prevent overfull lines
\setcounter{secnumdepth}{5}
% Make \paragraph and \subparagraph free-standing
\makeatletter
\ifx\paragraph\undefined\else
  \let\oldparagraph\paragraph
  \renewcommand{\paragraph}{
    \@ifstar
      \xxxParagraphStar
      \xxxParagraphNoStar
  }
  \newcommand{\xxxParagraphStar}[1]{\oldparagraph*{#1}\mbox{}}
  \newcommand{\xxxParagraphNoStar}[1]{\oldparagraph{#1}\mbox{}}
\fi
\ifx\subparagraph\undefined\else
  \let\oldsubparagraph\subparagraph
  \renewcommand{\subparagraph}{
    \@ifstar
      \xxxSubParagraphStar
      \xxxSubParagraphNoStar
  }
  \newcommand{\xxxSubParagraphStar}[1]{\oldsubparagraph*{#1}\mbox{}}
  \newcommand{\xxxSubParagraphNoStar}[1]{\oldsubparagraph{#1}\mbox{}}
\fi
\makeatother

\usepackage{color}
\usepackage{fancyvrb}
\newcommand{\VerbBar}{|}
\newcommand{\VERB}{\Verb[commandchars=\\\{\}]}
\DefineVerbatimEnvironment{Highlighting}{Verbatim}{commandchars=\\\{\}}
% Add ',fontsize=\small' for more characters per line
\usepackage{framed}
\definecolor{shadecolor}{RGB}{241,243,245}
\newenvironment{Shaded}{\begin{snugshade}}{\end{snugshade}}
\newcommand{\AlertTok}[1]{\textcolor[rgb]{0.68,0.00,0.00}{#1}}
\newcommand{\AnnotationTok}[1]{\textcolor[rgb]{0.37,0.37,0.37}{#1}}
\newcommand{\AttributeTok}[1]{\textcolor[rgb]{0.40,0.45,0.13}{#1}}
\newcommand{\BaseNTok}[1]{\textcolor[rgb]{0.68,0.00,0.00}{#1}}
\newcommand{\BuiltInTok}[1]{\textcolor[rgb]{0.00,0.23,0.31}{#1}}
\newcommand{\CharTok}[1]{\textcolor[rgb]{0.13,0.47,0.30}{#1}}
\newcommand{\CommentTok}[1]{\textcolor[rgb]{0.37,0.37,0.37}{#1}}
\newcommand{\CommentVarTok}[1]{\textcolor[rgb]{0.37,0.37,0.37}{\textit{#1}}}
\newcommand{\ConstantTok}[1]{\textcolor[rgb]{0.56,0.35,0.01}{#1}}
\newcommand{\ControlFlowTok}[1]{\textcolor[rgb]{0.00,0.23,0.31}{\textbf{#1}}}
\newcommand{\DataTypeTok}[1]{\textcolor[rgb]{0.68,0.00,0.00}{#1}}
\newcommand{\DecValTok}[1]{\textcolor[rgb]{0.68,0.00,0.00}{#1}}
\newcommand{\DocumentationTok}[1]{\textcolor[rgb]{0.37,0.37,0.37}{\textit{#1}}}
\newcommand{\ErrorTok}[1]{\textcolor[rgb]{0.68,0.00,0.00}{#1}}
\newcommand{\ExtensionTok}[1]{\textcolor[rgb]{0.00,0.23,0.31}{#1}}
\newcommand{\FloatTok}[1]{\textcolor[rgb]{0.68,0.00,0.00}{#1}}
\newcommand{\FunctionTok}[1]{\textcolor[rgb]{0.28,0.35,0.67}{#1}}
\newcommand{\ImportTok}[1]{\textcolor[rgb]{0.00,0.46,0.62}{#1}}
\newcommand{\InformationTok}[1]{\textcolor[rgb]{0.37,0.37,0.37}{#1}}
\newcommand{\KeywordTok}[1]{\textcolor[rgb]{0.00,0.23,0.31}{\textbf{#1}}}
\newcommand{\NormalTok}[1]{\textcolor[rgb]{0.00,0.23,0.31}{#1}}
\newcommand{\OperatorTok}[1]{\textcolor[rgb]{0.37,0.37,0.37}{#1}}
\newcommand{\OtherTok}[1]{\textcolor[rgb]{0.00,0.23,0.31}{#1}}
\newcommand{\PreprocessorTok}[1]{\textcolor[rgb]{0.68,0.00,0.00}{#1}}
\newcommand{\RegionMarkerTok}[1]{\textcolor[rgb]{0.00,0.23,0.31}{#1}}
\newcommand{\SpecialCharTok}[1]{\textcolor[rgb]{0.37,0.37,0.37}{#1}}
\newcommand{\SpecialStringTok}[1]{\textcolor[rgb]{0.13,0.47,0.30}{#1}}
\newcommand{\StringTok}[1]{\textcolor[rgb]{0.13,0.47,0.30}{#1}}
\newcommand{\VariableTok}[1]{\textcolor[rgb]{0.07,0.07,0.07}{#1}}
\newcommand{\VerbatimStringTok}[1]{\textcolor[rgb]{0.13,0.47,0.30}{#1}}
\newcommand{\WarningTok}[1]{\textcolor[rgb]{0.37,0.37,0.37}{\textit{#1}}}

\providecommand{\tightlist}{%
  \setlength{\itemsep}{0pt}\setlength{\parskip}{0pt}}\usepackage{longtable,booktabs,array}
\usepackage{calc} % for calculating minipage widths
% Correct order of tables after \paragraph or \subparagraph
\usepackage{etoolbox}
\makeatletter
\patchcmd\longtable{\par}{\if@noskipsec\mbox{}\fi\par}{}{}
\makeatother
% Allow footnotes in longtable head/foot
\IfFileExists{footnotehyper.sty}{\usepackage{footnotehyper}}{\usepackage{footnote}}
\makesavenoteenv{longtable}
\usepackage{graphicx}
\makeatletter
\def\maxwidth{\ifdim\Gin@nat@width>\linewidth\linewidth\else\Gin@nat@width\fi}
\def\maxheight{\ifdim\Gin@nat@height>\textheight\textheight\else\Gin@nat@height\fi}
\makeatother
% Scale images if necessary, so that they will not overflow the page
% margins by default, and it is still possible to overwrite the defaults
% using explicit options in \includegraphics[width, height, ...]{}
\setkeys{Gin}{width=\maxwidth,height=\maxheight,keepaspectratio}
% Set default figure placement to htbp
\makeatletter
\def\fps@figure{htbp}
\makeatother

\usepackage{fvextra}
\DefineVerbatimEnvironment{Highlighting}{Verbatim}{breaklines,commandchars=\\\{\}}
\usepackage{booktabs}
\usepackage{graphicx}
\usepackage{fancyhdr}
\usepackage{xcolor}
\definecolor{codebgcolor}{rgb}{0.95,0.95,0.95}
\pagestyle{fancy}
\fancyhead[L]{Spatial Statistics Notes}
\fancyhead[R]{\thepage}
\renewcommand{\headrulewidth}{0.4pt}
\renewcommand{\footrulewidth}{0.4pt}
\makeatletter
\@ifpackageloaded{caption}{}{\usepackage{caption}}
\AtBeginDocument{%
\ifdefined\contentsname
  \renewcommand*\contentsname{Table of contents}
\else
  \newcommand\contentsname{Table of contents}
\fi
\ifdefined\listfigurename
  \renewcommand*\listfigurename{List of Figures}
\else
  \newcommand\listfigurename{List of Figures}
\fi
\ifdefined\listtablename
  \renewcommand*\listtablename{List of Tables}
\else
  \newcommand\listtablename{List of Tables}
\fi
\ifdefined\figurename
  \renewcommand*\figurename{Figure}
\else
  \newcommand\figurename{Figure}
\fi
\ifdefined\tablename
  \renewcommand*\tablename{Table}
\else
  \newcommand\tablename{Table}
\fi
}
\@ifpackageloaded{float}{}{\usepackage{float}}
\floatstyle{ruled}
\@ifundefined{c@chapter}{\newfloat{codelisting}{h}{lop}}{\newfloat{codelisting}{h}{lop}[chapter]}
\floatname{codelisting}{Listing}
\newcommand*\listoflistings{\listof{codelisting}{List of Listings}}
\makeatother
\makeatletter
\makeatother
\makeatletter
\@ifpackageloaded{caption}{}{\usepackage{caption}}
\@ifpackageloaded{subcaption}{}{\usepackage{subcaption}}
\makeatother
\makeatletter
\@ifpackageloaded{tcolorbox}{}{\usepackage[skins,breakable]{tcolorbox}}
\makeatother
\makeatletter
\@ifundefined{shadecolor}{\definecolor{shadecolor}{rgb}{.97, .97, .97}}{}
\makeatother
\makeatletter
\@ifundefined{codebgcolor}{\definecolor{codebgcolor}{named}{codebgcolor}}{}
\makeatother
\makeatletter
\ifdefined\Shaded\renewenvironment{Shaded}{\begin{tcolorbox}[sharp corners, colback={codebgcolor}, borderline west={3pt}{0pt}{shadecolor}, boxrule=0pt, enhanced, breakable, frame hidden]}{\end{tcolorbox}}\fi
\makeatother

\ifLuaTeX
  \usepackage{selnolig}  % disable illegal ligatures
\fi
\usepackage{bookmark}

\IfFileExists{xurl.sty}{\usepackage{xurl}}{} % add URL line breaks if available
\urlstyle{same} % disable monospaced font for URLs
\hypersetup{
  pdftitle={Spatial Statistics Notes},
  pdfauthor={Brian Cervantes Alvarez},
  pdfkeywords={Spatial Statistics, Kriging, Variograms, SAR
Models, NNGP},
  colorlinks=true,
  linkcolor={blue},
  filecolor={Maroon},
  citecolor={Blue},
  urlcolor={Blue},
  pdfcreator={LaTeX via pandoc}}


\title{Spatial Statistics Notes}
\author{Brian Cervantes Alvarez}
\date{2025-02-04}

\begin{document}
\maketitle
\begin{abstract}
Spatial statistics is a crucial field that focuses on analyzing
spatially referenced data, incorporating geographic and locational
information to model relationships and dependencies. These notes cover
foundational concepts such as spatial data types, coordinate reference
systems, and spatial dependence, along with advanced topics including
geostatistical models, spatial regression, and simultaneous
autoregressive (SAR) models. Key topics include the multivariate normal
distribution, variograms, kriging techniques, and methods for handling
spatially structured big data, such as Nearest-Neighbor Gaussian
Processes (NNGP). The notes also delve into areal data analysis, spatial
autocorrelation measures like Moran's I and Geary's C, and model-based
approaches for predicting spatial outcomes. A strong emphasis is placed
on practical implementation, using R packages such as \texttt{sp},
\texttt{sf}, \texttt{gstat}, \texttt{spmodel}, and \texttt{spNNGP} for
real-world spatial data analysis. Applications span environmental
monitoring, urban planning, epidemiology, and more, demonstrating the
critical role of spatial statistics in decision-making and scientific
research.
\end{abstract}

\renewcommand*\contentsname{Table of contents}
{
\hypersetup{linkcolor=}
\setcounter{tocdepth}{2}
\tableofcontents
}

\setstretch{2}
\chapter{Introduction to Spatial
Statistics}\label{introduction-to-spatial-statistics}

\section{Preliminaries}\label{preliminaries}

\subsection{Tobler's First Law of
Geography}\label{toblers-first-law-of-geography}

``Everything is related to everything else, but near things are more
related than distant things.'' --- Tobler (1970)

\textbf{Why does this matter?} Imagine you're studying temperature
changes across a region. If two locations are close, their temperatures
are likely more similar than two locations far apart. This principle is
fundamental in spatial analysis, affecting how we model relationships
between data points.

\subsection{Types of Spatial Data}\label{types-of-spatial-data}

\begin{enumerate}
\def\labelenumi{\arabic{enumi}.}
\tightlist
\item
  \textbf{Geostatistical Data}: Continuous spatial variables, e.g., soil
  pollution measurements.
\item
  \textbf{Areal (Lattice) Data}: Aggregated over regions, e.g., census
  data by county.
\item
  \textbf{Point Pattern Data}: Locations of events, e.g., earthquake
  occurrences.
\end{enumerate}

\textbf{Example:} Consider air quality measurements across a city. If
measured at scattered sensor locations, it's geostatistical data. If
summarized at the neighborhood level, it's areal data. If recording only
pollution source locations, it's point pattern data.

\section{Spatial Coordinate Systems}\label{spatial-coordinate-systems}

A coordinate system provides a mathematical way to specify locations.
Important terms:

\begin{itemize}
\tightlist
\item
  \textbf{Datum}: Defines the origin and scale of a coordinate system.
\item
  \textbf{Geodetic Datum}: Relates the coordinate system to Earth's
  shape.
\item
  \textbf{Coordinate Reference System (CRS)}: Specifies how a dataset's
  coordinates relate to the real world.
\end{itemize}

\textbf{Analogy:} Think of CRS like different map apps---Google Maps,
Apple Maps, and OpenStreetMap all display the same Earth, but their
coordinate systems may differ slightly.

\textbf{Example:} The Universal Transverse Mercator (UTM) system divides
the world into zones, making local measurements more accurate.

\chapter{Multivariate Normal
Distributions}\label{multivariate-normal-distributions}

\section{Bivariate Normal
Distribution}\label{bivariate-normal-distribution}

A two-variable generalization of the normal distribution, described by:

\begin{itemize}
\tightlist
\item
  \textbf{Means (\(\mu_1, \mu_2\))}: Expected values.
\item
  \textbf{Variances (\(\sigma_1^2, \sigma_2^2\))}: Spread of each
  variable.
\item
  \textbf{Correlation (\(\rho\))}: Strength of their relationship.
\end{itemize}

\textbf{Why does this matter?} Many spatial models assume normality,
making this distribution crucial for prediction and inference.

\textbf{Example:} If you measure temperature and humidity across a city,
they likely follow a bivariate normal distribution, where hotter days
tend to be more humid.

\section{Variance-Covariance
Matrices}\label{variance-covariance-matrices}

For multiple variables, we use a variance-covariance matrix: \[
\Sigma = \begin{bmatrix} \sigma_1^2 & \text{cov}(Y_1, Y_2) \\ \text{cov}(Y_1, Y_2) & \sigma_2^2 \end{bmatrix}
\] \textbf{Properties:}

\begin{itemize}
\tightlist
\item
  Always symmetric (\(\text{cov}(Y_1, Y_2) = \text{cov}(Y_2, Y_1)\)).
\item
  Positive definite (ensures meaningful variability).
\end{itemize}

\textbf{Analogy:} Think of covariance like how two stocks move
together---if one rises when the other does, their covariance is
positive.

\section{Multivariate Normal
Distribution}\label{multivariate-normal-distribution}

A generalization for multiple correlated variables, defined by a mean
vector and covariance matrix: \[
f(y) = (2\pi)^{-n/2} |\Sigma|^{-1/2} \exp \left( -\frac{1}{2} (y - \mu)' \Sigma^{-1} (y - \mu) \right)
\]

\textbf{Example:} If you analyze rainfall, temperature, and humidity
together, they likely follow a multivariate normal distribution, helping
in climate modeling and weather prediction.

\chapter{More on Variograms; Spatial Prediction With
Covariates}\label{more-on-variograms-spatial-prediction-with-covariates}

\section{Practical Range and Matérn Variogram
Model}\label{practical-range-and-matuxe9rn-variogram-model}

The \textbf{practical range} is the distance at which covariance drops
to 5\% of the partial sill, varying by model type.

\textbf{Why does this matter?} It helps determine the spatial scale of
correlation, guiding interpolation decisions.

\textbf{Example:} In environmental science, it informs sensor placement
for air pollution monitoring.

\subsection{Matérn Variogram Model}\label{matuxe9rn-variogram-model}

The Matérn model introduces flexibility via the \textbf{smoothness
parameter (κ)}, controlling how covariance decays.

\textbf{Analogy:} Think of κ like road conditions---higher values make
the transition from smooth to rough more gradual.

\section{Universal Kriging}\label{universal-kriging}

\textbf{Universal kriging} extends ordinary kriging by incorporating
covariates like elevation.

\textbf{Why does this matter?} It allows for better predictions when
spatial trends exist.

\textbf{Example:} Predicting soil contamination based on both spatial
location and land elevation.

\chapter{Regression With Geostatistical
Data}\label{regression-with-geostatistical-data}

\section{Prediction vs.~Estimation}\label{prediction-vs.-estimation}

\begin{itemize}
\tightlist
\item
  \textbf{Prediction}: Forecasts values at unsampled locations.
\item
  \textbf{Estimation}: Determines underlying relationships between
  variables.
\end{itemize}

\textbf{Example:} Predicting temperature at an unmeasured site versus
estimating how temperature depends on elevation.

\section{Spatial Regression}\label{spatial-regression}

Spatial regression extends traditional models by accounting for spatial
dependence.

\subsection{Key Model Differences:}\label{key-model-differences}

\begin{enumerate}
\def\labelenumi{\arabic{enumi}.}
\tightlist
\item
  \textbf{Ordinary Regression} assumes independence.
\item
  \textbf{Spatial Regression} incorporates a spatial covariance
  structure.
\end{enumerate}

\textbf{Example:} Modeling heavy metal contamination while accounting
for proximity effects.

\subsection{Practical Implementation}\label{practical-implementation}

\begin{itemize}
\tightlist
\item
  \texttt{lm()} for independent models.
\item
  \texttt{splm()} for spatial models, estimating covariance parameters.
\end{itemize}

\textbf{Why does this matter?} Spatial regression reduces bias and
improves inference in geographically structured data.

\chapter{Simultaneous Autoregressive (SAR)
Models}\label{simultaneous-autoregressive-sar-models}

\section{Row-Standardized Weights}\label{row-standardized-weights}

SAR models rely on spatial weight matrices to define relationships
between spatial units.

\textbf{Example:} Consider housing prices in a city---neighboring
properties often influence each other. SAR models capture this
dependency through a structured matrix.

\textbf{Types of Contiguity:}

\begin{itemize}
\tightlist
\item
  \textbf{Queen contiguity:} A single shared boundary point defines a
  neighborhood.
\item
  \textbf{Binary weights:} Assigns 1 if two areas share a boundary and 0
  otherwise.
\item
  \textbf{Row-standardized weights:} Adjusts for different numbers of
  neighbors to ensure interpretability.
\end{itemize}

\textbf{Analogy:} Think of binary weights like direct friendships,
whereas row-standardized weights are like adjusting for
popularity---someone with many friends has each connection weighted
less.

\section{Simultaneous Autoregressive (SAR)
Models}\label{simultaneous-autoregressive-sar-models-1}

SAR models incorporate spatial dependence by modifying regression
equations: \[
Z = X\beta + B(Z - X\beta) + d
\] where:

\begin{itemize}
\tightlist
\item
  \(B = \phi W\) describes spatial relationships,
\item
  \(W\) is the spatial weights matrix,
\item
  \(d\) is the error term.
\end{itemize}

\subsection{Variance-Covariance
Structure}\label{variance-covariance-structure}

The SAR model leads to: \[
\Sigma_{SAR} = \sigma^2_{de} [(I - \phi W)(I - \phi W)^T]^{-1}
\] This accounts for spatial correlation in the dependent variable.

\textbf{Why does this matter?} Ordinary regression assumes independence,
but ignoring spatial dependence can lead to biased estimates. SAR models
correct this.

\section{Implementing SAR Models in
R}\label{implementing-sar-models-in-r}

Using the \texttt{spmodel} package:

\begin{Shaded}
\begin{Highlighting}[]
\FunctionTok{library}\NormalTok{(spmodel)}
\NormalTok{boston\_sar }\OtherTok{\textless{}{-}} \FunctionTok{spautor}\NormalTok{(MEDV }\SpecialCharTok{\textasciitilde{}} \DecValTok{1}\NormalTok{, }\AttributeTok{data =}\NormalTok{ boston\_sf, }\AttributeTok{spcov\_type =} \StringTok{"sar"}\NormalTok{, }\AttributeTok{row\_s =} \ConstantTok{FALSE}\NormalTok{)}
\FunctionTok{summary}\NormalTok{(boston\_sar)}
\end{Highlighting}
\end{Shaded}

\textbf{Interpretation}

\begin{itemize}
\tightlist
\item
  \textbf{Intercept:} Represents the average property value.
\item
  \textbf{SAR spatial covariance parameters:} Indicate spatial
  dependence strength.
\end{itemize}

\section{Row-Standardization and Its
Effect}\label{row-standardization-and-its-effect}

Row-standardization ensures each row sums to 1, making comparisons fair
across regions:

\begin{Shaded}
\begin{Highlighting}[]
\NormalTok{boston\_sar }\OtherTok{\textless{}{-}} \FunctionTok{spautor}\NormalTok{(MEDV }\SpecialCharTok{\textasciitilde{}} \DecValTok{1}\NormalTok{, }\AttributeTok{data =}\NormalTok{ boston\_sf, }\AttributeTok{spcov\_type =} \StringTok{"sar"}\NormalTok{, }\AttributeTok{row\_s =} \ConstantTok{TRUE}\NormalTok{)}
\FunctionTok{summary}\NormalTok{(boston\_sar)}
\end{Highlighting}
\end{Shaded}

\textbf{Example:} The impact of neighbors is scaled relative to the
number of connections, preventing dominant effects from highly connected
regions.

\textbf{Analogy:} Think of row-standardization like adjusting for social
media influence---someone with 1,000 friends has less influence per
friend compared to someone with just 10.

\textbf{Why does this matter?} Row-standardization helps maintain
stability in the model, particularly for regions with varying numbers of
neighbors.

\chapter{Additional Topics}\label{additional-topics}

\section{Spatial Point Processes}\label{spatial-point-processes}

Spatial point processes model the random occurrence of events in space.
They're crucial when the data consist of discrete event
locations---think tracking the spread of disease outbreaks or modeling
the locations of trees in a forest.

\subsection{Poisson Point Process}\label{poisson-point-process}

The \textbf{homogeneous Poisson point process} assumes that events occur
completely at random with a constant intensity \(\lambda\) over the
study region. The number of events in a region of area \(A\) follows a
Poisson distribution: \[
P(N(A) = k) = \frac{(\lambda A)^k e^{-\lambda A}}{k!}
\]

\textbf{Example:} If you're tracking meteorite impacts over a vast area,
you might assume they occur randomly following a Poisson process.

\subsubsection{R Implementation}\label{r-implementation}

\begin{Shaded}
\begin{Highlighting}[]
\FunctionTok{library}\NormalTok{(spatstat)}
\CommentTok{\# Define a rectangular window for the study area}
\NormalTok{win }\OtherTok{\textless{}{-}} \FunctionTok{owin}\NormalTok{(}\AttributeTok{xrange =} \FunctionTok{c}\NormalTok{(}\DecValTok{0}\NormalTok{, }\DecValTok{100}\NormalTok{), }\AttributeTok{yrange =} \FunctionTok{c}\NormalTok{(}\DecValTok{0}\NormalTok{, }\DecValTok{100}\NormalTok{))}
\CommentTok{\# Simulate a homogeneous Poisson process with intensity 0.05 events per unit area}
\NormalTok{pp\_poisson }\OtherTok{\textless{}{-}} \FunctionTok{rpoispp}\NormalTok{(}\AttributeTok{lambda =} \FloatTok{0.05}\NormalTok{, }\AttributeTok{win =}\NormalTok{ win)}
\FunctionTok{plot}\NormalTok{(pp\_poisson, }\AttributeTok{main =} \StringTok{"Homogeneous Poisson Point Process"}\NormalTok{)}
\end{Highlighting}
\end{Shaded}

\subsection{Cluster and Inhibition
Processes}\label{cluster-and-inhibition-processes}

\begin{itemize}
\tightlist
\item
  \textbf{Cluster Processes:} Events tend to occur in groups. For
  instance, disease cases might cluster in certain neighborhoods due to
  environmental factors.
\item
  \textbf{Inhibition Processes:} Events tend to repel each other (e.g.,
  trees that compete for resources).
\end{itemize}

\textbf{Analogy:} Imagine a party---people might cluster around a snack
table (cluster process) or spread out on the dance floor to give
everyone room (inhibition process).

\section{Bayesian Spatial Models}\label{bayesian-spatial-models}

Bayesian approaches allow you to incorporate prior knowledge into
spatial models, which is especially useful in data-scarce or highly
uncertain environments.

\subsection{Hierarchical Bayesian Spatial
Models}\label{hierarchical-bayesian-spatial-models}

In a hierarchical setup, spatial data are modeled on multiple levels.
For example, the observed data might depend on latent spatial random
effects that capture unmeasured spatial variation: \[
Y(s) = X(s)\beta + w(s) + \epsilon(s)
\] where: - \(w(s)\) is a latent spatial process (often modeled as a
Gaussian Process), - \(\epsilon(s)\) is independent error.

\textbf{Why does this matter?} By explicitly modeling spatial random
effects, you capture dependencies that traditional models might miss,
leading to improved uncertainty quantification.

\subsubsection{\texorpdfstring{R Implementation with
\texttt{spBayes}}{R Implementation with spBayes}}\label{r-implementation-with-spbayes}

\begin{Shaded}
\begin{Highlighting}[]
\FunctionTok{library}\NormalTok{(spBayes)}
\CommentTok{\# Assume we have spatial data with coordinates \textquotesingle{}coords\textquotesingle{} and response \textquotesingle{}Y\textquotesingle{}}
\CommentTok{\# Define a spatial decay parameter and variance components}
\NormalTok{starting }\OtherTok{\textless{}{-}} \FunctionTok{list}\NormalTok{(}\StringTok{"phi"} \OtherTok{=} \DecValTok{3}\NormalTok{, }\StringTok{"sigma.sq"} \OtherTok{=} \DecValTok{1}\NormalTok{, }\StringTok{"tau.sq"} \OtherTok{=} \FloatTok{0.1}\NormalTok{)}
\NormalTok{tuning   }\OtherTok{\textless{}{-}} \FunctionTok{list}\NormalTok{(}\StringTok{"phi"} \OtherTok{=} \FloatTok{0.1}\NormalTok{, }\StringTok{"sigma.sq"} \OtherTok{=} \FloatTok{0.1}\NormalTok{, }\StringTok{"tau.sq"} \OtherTok{=} \FloatTok{0.05}\NormalTok{)}
\NormalTok{priors   }\OtherTok{\textless{}{-}} \FunctionTok{list}\NormalTok{(}\StringTok{"phi.Unif"} \OtherTok{=} \FunctionTok{c}\NormalTok{(}\FloatTok{0.1}\NormalTok{, }\DecValTok{10}\NormalTok{), }\StringTok{"sigma.sq.IG"} \OtherTok{=} \FunctionTok{c}\NormalTok{(}\DecValTok{2}\NormalTok{, }\DecValTok{1}\NormalTok{), }\StringTok{"tau.sq.IG"} \OtherTok{=} \FunctionTok{c}\NormalTok{(}\DecValTok{2}\NormalTok{, }\FloatTok{0.1}\NormalTok{))}
\CommentTok{\# Fit a simple spatial regression model}
\NormalTok{bayes\_model }\OtherTok{\textless{}{-}} \FunctionTok{spLM}\NormalTok{(Y }\SpecialCharTok{\textasciitilde{}}\NormalTok{ X1 }\SpecialCharTok{+}\NormalTok{ X2, }
                    \AttributeTok{data =}\NormalTok{ your\_data, }
                    \AttributeTok{coords =}\NormalTok{ your\_data[, }\FunctionTok{c}\NormalTok{(}\StringTok{"x"}\NormalTok{, }\StringTok{"y"}\NormalTok{)],}
                    \AttributeTok{starting =}\NormalTok{ starting, }
                    \AttributeTok{tuning =}\NormalTok{ tuning, }
                    \AttributeTok{priors =}\NormalTok{ priors, }
                    \AttributeTok{cov.model =} \StringTok{"exponential"}\NormalTok{, }
                    \AttributeTok{n.samples =} \DecValTok{1000}\NormalTok{)}
\FunctionTok{summary}\NormalTok{(bayes\_model)}
\end{Highlighting}
\end{Shaded}

\textbf{Analogy:} Think of Bayesian models as using ``wisdom from the
past'' (priors) to help guide predictions when new data are sparse or
noisy.

\section{Nearest-Neighbor Gaussian Processes
(NNGP)}\label{nearest-neighbor-gaussian-processes-nngp}

For large spatial datasets, traditional Gaussian Process models can be
computationally prohibitive because of the inversion of large covariance
matrices. \textbf{NNGP} approximates the full Gaussian Process by
assuming that each location is conditionally independent given its
nearest neighbors.

\subsection{Key Benefits}\label{key-benefits}

\begin{itemize}
\tightlist
\item
  \textbf{Scalability:} Efficiently handles datasets with thousands (or
  more) spatial locations.
\item
  \textbf{Flexibility:} Retains much of the interpretability of full
  Gaussian Processes while reducing computational overhead.
\end{itemize}

\textbf{Example:} In urban planning, when analyzing city-wide sensor
data (e.g., air quality), NNGP allows you to process massive datasets
quickly without sacrificing much accuracy.

\subsubsection{\texorpdfstring{R Implementation with
\texttt{spNNGP}}{R Implementation with spNNGP}}\label{r-implementation-with-spnngp}

\begin{Shaded}
\begin{Highlighting}[]
\FunctionTok{library}\NormalTok{(spNNGP)}
\CommentTok{\# Assume we have data \textquotesingle{}your\_data\textquotesingle{} with coordinates and response \textquotesingle{}Y\textquotesingle{}}
\NormalTok{coords }\OtherTok{\textless{}{-}} \FunctionTok{as.matrix}\NormalTok{(your\_data[, }\FunctionTok{c}\NormalTok{(}\StringTok{"x"}\NormalTok{, }\StringTok{"y"}\NormalTok{)])}
\CommentTok{\# Define priors and starting values for the model parameters}
\NormalTok{starting }\OtherTok{\textless{}{-}} \FunctionTok{list}\NormalTok{(}\StringTok{"phi"} \OtherTok{=} \DecValTok{3}\NormalTok{, }\StringTok{"sigma.sq"} \OtherTok{=} \DecValTok{1}\NormalTok{, }\StringTok{"tau.sq"} \OtherTok{=} \FloatTok{0.1}\NormalTok{)}
\NormalTok{priors   }\OtherTok{\textless{}{-}} \FunctionTok{list}\NormalTok{(}\StringTok{"phi.Unif"} \OtherTok{=} \FunctionTok{c}\NormalTok{(}\FloatTok{0.1}\NormalTok{, }\DecValTok{10}\NormalTok{), }\StringTok{"sigma.sq.IG"} \OtherTok{=} \FunctionTok{c}\NormalTok{(}\DecValTok{2}\NormalTok{, }\DecValTok{1}\NormalTok{), }\StringTok{"tau.sq.IG"} \OtherTok{=} \FunctionTok{c}\NormalTok{(}\DecValTok{2}\NormalTok{, }\FloatTok{0.1}\NormalTok{))}
\CommentTok{\# Fit the NNGP model using 10 nearest neighbors}
\NormalTok{nngp\_model }\OtherTok{\textless{}{-}} \FunctionTok{spNNGP}\NormalTok{(Y }\SpecialCharTok{\textasciitilde{}}\NormalTok{ X1 }\SpecialCharTok{+}\NormalTok{ X2, }
                     \AttributeTok{data =}\NormalTok{ your\_data, }
                     \AttributeTok{coords =}\NormalTok{ coords,}
                     \AttributeTok{starting =}\NormalTok{ starting, }
                     \AttributeTok{priors =}\NormalTok{ priors, }
                     \AttributeTok{n.neighbors =} \DecValTok{10}\NormalTok{,}
                     \AttributeTok{cov.model =} \StringTok{"exponential"}\NormalTok{, }
                     \AttributeTok{n.samples =} \DecValTok{1000}\NormalTok{, }
                     \AttributeTok{n.threads =} \DecValTok{2}\NormalTok{)}
\FunctionTok{summary}\NormalTok{(nngp\_model)}
\end{Highlighting}
\end{Shaded}

\textbf{Analogy:} Think of NNGP like a squad of Gen Z influencers---each
location only ``listens'' to its closest peers rather than the entire
network, making the model both trendy and computationally sleek.




\end{document}
